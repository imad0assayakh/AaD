%%This is a very basic article template.
%%There is just one section and two subsections.



\documentclass{article}
\usepackage[T1]{fontenc}
\usepackage[french]{babel}
\begin{document}

\section{Objectif : Responsable commercial}
Le responsable commercial cherche � �quilibrer le nombre
d'unit�s de ${A, B, C}$ (famille 1) et ${D, E, F}$ (famille 2) afin que ces deux
familles contiennent le m�me nombre d'unit�s (� $\epsilon$ unit�(s) pr�s).
~\\
En prennant en compte les contraintes d�finies avant et en ajoutant une nouvelle
contrainte d'�quilibre entre les deux familles de produits on se rend compte que
la meilleure solution sera de ne rien produire. C'est solution n'est pas
avantageuse. On introduit par con�quent une autre contrainte - nous allons 
essayer d'�quilibrer les familles de produits tout en conservant un b�n�fice maximal.

\subsection{Mod�lisation}
Soient :
 \begin{itemize}
%   \item $N_1$ le nombre de produits de la famille 1 fabriqués.
%   \item $N_2$ le nombre de produits de la famille 2 fabriqués.
	\item $n_A$ le nombre de produits A usin�s. 
	\item $n_B$ le nombre de produits B usin�s. 
	\item $n_C$ le nombre de produits C usin�s. 
	\item $n_D$ le nombre de produits D usin�s. 
	\item $n_E$ le nombre de produits E usin�s. 
	\item $n_F$ le nombre de produits F usin�s. 
\end{itemize}
~\\
Pour �tudier l'�quilibre entre les deux familles des produits on d�finira une
fonction qui sera �gale � la diff�rence entre les quantit�s des produits
provenant des deux familles, soit: 

\begin{displaymath}
F = (n_A+n_B+n_C)-(n_D+n_E+n_F)
\end{displaymath}

La contrainte sur l'�quilibre va se traduire par l'essai de minimiser la
fonction F, c'est � dire:
\begin{eqnarray*}
	|F| &\leq& \epsilon\\
	\Leftrightarrow -\epsilon \leq F &\leq& \epsilon\\
	\Leftrightarrow -\epsilon \leq (n_A+n_B+n_C)-(n_D+n_E+n_F)
	&\leq& \epsilon\\
	Avec& \epsilon \rightarrow 0.\\
\end{eqnarray*} 
~\\
Ensuite on va essayer de maximiser le b�n�fice obtenu. 
On va proc�der de la mani�re suivante:  
\begin{itemize}
   \item Nous fixons un certain b�n�fice � atteindre, par
	     exemple 60\% du b�n�fice maximal.
   \item Pour le b�n�fice choisi, nous essayons de minimiser F. 
\end{itemize}
~\\
En r�p�tant cette d�marche un certain nombre de fois, nous pouvons tracer une
courbe repr�sentant la valeur de F en fonction du b�n�fice atteint. 
L'interpr�tation de cette courbe nous permettra de trouver le meilleur
compromis entre nos deux objectifs.

\subsection{D�cisions}

\subsection{Another subtitle}

More plain text.


\end{document}
