\newpage
\section{Objectif : Responsable commercial}
Le responsable commercial cherche à équilibrer le nombre
d'unités de ${A, B, C}$ (famille 1) et ${D, E, F}$ (famille 2) afin que ces deux
familles contiennent le même nombre d'unités ( à $\epsilon$
unité(s) près).\\
Autrement dit, l'écart entre le nombre d'unités produite pour la famille A et la famille B doit être inférieur à un seuil $\epsilon$.

\subsection{Modélisation}
Soient :
\begin{itemize}
  \item $N_1$ le nombre de produits de la famille 1 fabriqués.
  \item $N_2$ le nombre de produits de la famille 2 fabriqués.
\end{itemize}

\begin{eqnarray*}
	|N_1 - N_2| &\leq& \epsilon\\
	\Leftrightarrow -\epsilon \leq N_1 - N_2 &\leq& \epsilon\\
	\Leftrightarrow -\epsilon \leq \sum_{i = A}^{C} n_i - \sum_{j = D}^{F} n_j
	&\leq& \epsilon\\
\end{eqnarray*} 
Par concéquent, c'est cette nouvelle contrainte qui, venant s'ajouter aux
contraintes précédentes, va permettre de calculer le nombre d'unités A, B, C,
D, E, et F à fabriquer afin d'équilibrer les deux familles.\\
~\\
Nous obtenons la matrice suivante :
\begin{displaymath}
M = \left(
\begin{array}{cccccc}
1 & 1 & 1 & 1 & 1 & 1\\
\end{array}
\right)
\end{displaymath}
La matrice, très simple, représente la somme des différents produits.\\
La matrice des contraintes devient quant à elle :
INSÉRER MATRICE A MODIFIEE

\subsection{Décisions}

\subsection{Interprétation}
Evidemment, toutes les solutions triviales du type :
\begin{displaymath}
M_p = \left(
\begin{array}{cccccc}
N \pm \epsilon & N \pm \epsilon & N \pm \epsilon & N & N & N\\
\end{array}
\right)
\end{displaymath}
ou encore 
\begin{displaymath}
M_p = \left(
\begin{array}{cccccc}
N & N & N & N \pm \epsilon & N \pm \epsilon & N \pm \epsilon\\
\end{array}
\right)
\end{displaymath}
\begin{center}
\textsl{Où $M_p$ est la matrice du nombre de produit, et $N \in \mathbbm{N}$}
\end{center}
sont des solutions \emph{valables}.\\
~\\
Ceci met en évidence qu'avec les critères définis plus haut, il n'y a pas de
solution plus \og valable\fg ~qu'une autre. Par concéquent nous pouvons :
\begin{itemize}
  \item Choisir une solution au hasard
  \item Augmenter le nombre de critère et notamment ceux en rapport avec les
  stocks disponibles, le prix des matières premières, ou encore le temps
  d'usinage nécessaire.
\end{itemize}
