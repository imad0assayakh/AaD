\newpage
\section{Objectif : Responsable d'atelier}
Le responsable d'atelier cherche à maximiser le nombre d'unités (toutes
catégories confondues) produites sous les contraintes définies précedemment.\\
Autrement dit, seul la quantité de matières premières disponible et le temps
maximum de travail limitera la production. Il n'intervient donc pour le moment
aucune notion de bénéfice.

\subsection{Modélisation}
Soit $N$ le nombre de produits fabriqués.

\begin{equation}
	N = \sum_{i = A}^{F} n_i
\end{equation} 
~\\
Par concéquent, nous obtenons la matrice suivante, qui représente la somme des
différents produits :  
\begin{displaymath}
M = \left(
\begin{array}{cccccc}
1 & 1 & 1 & 1 & 1 & 1\\
\end{array}
\right)
\end{displaymath}
~\\
La matrice des contraintes quant à elle, n'est pas modifiée.

\subsection{Décisions}
Instinctivement, on sait qu'il faudra fabriquer l'unité consommant le moins de
\og temps machine\fg et le moins de matières premières. On peux donc penser que
E et F seront les produits les plus fabriqués.\\
~\\
En pratique, on obtient les résultats suivants :\\

\begin{lstlisting}
N = linprog(Units, InfEqConstraints, InfEqValues);
\end{lstlisting}

\begin{displaymath}
M = \left(
\begin{array}{c}
0\\
56.732\\
38.6928\\
0\\
182.4608\\
98.9216 
\end{array}
\right)
\end{displaymath}
\begin{center}
\textbf{On peut donc, au maximum, fabriquer 376.8072 unités, tous produits
confondus.}\\
~\\
\textsl{Ceci confirme bien les résultats attendus, c'est à dire
qu'il faut donner la priorité aux produits consommant le moins de ressources.\\
On remarque cependant qu'il est préférable d'abandonner \textsl{A} et \textsl{C}
au profit d'autres.}
\end{center}
