\section{Stratégie du point de vue \textsl{comptable}}
\label{sec:comptable}
En tenant compte des coût de fonctionnement des machines, du coût d'achat des mati1ères premières,
du prix de vente des produits finis, et du temps d'usinage, le comptable de l'entreprise \textbf{Optim} cherche 
à maximiser les bénefices. Nous ne prendrons pas en compte la \og popularité \fg du produit ou encore
la répartition des unités produites les unes par rapport aux autres.

\subsection{Modélisation}
Soit $n_i$ le nombre de produit $i$ fabriqué.\\
Le coup fixe de production n'influant pas sur notre décision, nous considérerons uniquement le coût variable de
production.\\
~\\
Il est défini par la formule suivante:
\begin{displaymath}
	CV(i) = n_i \times \left (\sum_{j = 1}^{7} T_{i,j} \times 
	\frac{C_{i,j}}{60} + \sum_{k = 1}^{3} A_{k} \times Q_{k,i} \right )
\end{displaymath}
~\\
Le chiffre d'affaire par produit est :
\begin{displaymath}
	CA(i) = n_i \times V_i
\end{displaymath}
~\\
Par conséquent le bénefice par produit se calcule de la manière suivante :
\begin{eqnarray*}
	B(i)	&=& CA(i) - CV(i)\\
			&=& n_i  \times \left (V_i - \sum_{j = 1}^{7} T_{i,j} \times \frac{C_{i,j}}{60} +
	\sum_{k = 1}^{3} A_{k} \times Q_{k,i} \right )
\end{eqnarray*}
La matrice permettant de calculer, à partir du vecteur colonne $n$ du nombre de produits sortis de l'usine, le bénéfice
d'\emph{Optim} est donc la suivante :
\begin{eqnarray*}
	M_{B} &=&  \left(V - \left( \left(T \times C^{t} \times \frac{1}{60} \right)^{t} + \left(A \times Q\right) \right)\right) \\
		&\simeq& \begin{pmatrix} 6.0667 & 11.9833 & 12.4333 & 9.5333 & 31.6500 & 27.9000 \end{pmatrix}
\end{eqnarray*}

Remarquons qu'elle donne explicitement le bénéfice unitaire pour chaque produit. instincitvement, le produit E est 
le plus intéressant.

Nous chercherons à maximiser la fonction linéaire correspondant à cette matrice, donc (pour prendre une forme plus standard)
à minimiser son opposé.

\subsection{Stratégie adoptée}

En utilisant les outils Matlab, nous obtenons le résulat suivant :
\addCode{../SourcesMatlab/comptableSnippet.m}{matlab}
\[
	n_{optimal} = 
\begin{pmatrix}
    0.0000 \\
   20.4082 \\
    0.0000 \\
    0.0000 \\
  242.5000 \\
   94.1837
\end{pmatrix}
\]
\begin{center}
\fbox{\textbf{Du point de vue strictement comptable, le bénefice réalisable est 357.0919 \textcurrency.}}
\end{center}

\subsubsection{Interprétation}
\begin{itemize}
	\item Les produits (A), (C), (D) doivent être abandonnés.
	\item Le produit (E) doit être \og surproduit\fg (en 242,5 exemplaires), tout comme le produit (F), mais dans 
	une moindre mesure.
\end{itemize}
~\\
Ce résultat était (en partie) prévisible à partir de la matrice $M_{B}$, puisque le produit E est le plus rentable.
Il faudra donc que l'entreprise \textbf{Optim}, pour optimiser le bénéfice, en produise le plus possible tout en 
utilisant intelligemment les matières premières et ressources humaines restantes.
