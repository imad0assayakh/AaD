\newpage
\section{Stratégie du point de vue du \textsl{Responsable d'atelier}}
Le responsable d'atelier cherche à maximiser le nombre d'unités (toutes
catégories confondues) produites sous les contraintes définies précedemment.\\
Autrement dit, seul la quantité de matières premières disponibles et le temps
maximum de travail limitera la production.\\
\textbf{Il n'intervient donc pour le moment aucune notion de bénéfice.}

\subsection{Modélisation}
Soit $N$ le nombre total de produits fabriqués. Trivialement, c'est la somme du nombre d'unités (A), (B), (C), (D),
(E), et (F) produites :

\begin{equation}
	N = \sum_{i = A}^{F} n_i
\end{equation} 
~\\
Par concéquent la matrice suivante représente la somme des différents produits :  
\begin{displaymath}
	M = \left(
	\begin{array}{cccccc}
		1 & 1 & 1 & 1 & 1 & 1\\
	\end{array}
	\right)
\end{displaymath}
~\\
La matrice des contraintes quant à elle, n'est pas modifiée. En effet, le responsable d'atelier d'\textbf{Optim} n'ajoute 
aucune contrainte à la production.

\subsection{Stratégie adoptée}
Instinctivement, il est évident qu'il faudra fabriquer au maximum l'unité consommant le moins de
\og temps machine\fg et de matières premières.
On peux donc penser que (E) et (F) seront les produits les plus fabriqués.\\
~\\
En pratique, on obtient les résultats suivants :\\
\addCode{../SourcesMatlab/atelierSnippet.m}{matlab}

\begin{displaymath}
	M = \left(
	\begin{array}{c}
		0\\
		56.732\\
		38.6928\\
		0\\
		182.4608\\
		98.9216 
	\end{array}
	\right)
\end{displaymath}
\begin{center}
	\fbox{On peut donc, au maximum, fabriquer 376.8072 unités, tous produits
	confondus.}\\
	~\\
\end{center}

\subsubsection{Interprétation}
Encore une fois :
\begin{itemize}
	\item Les produits (A) et (D) doivent être abandonnés.
	\item Le produit (E) doit être \og surproduit\fg (en 242,5 exemplaires), tout comme le produit (F), mais dans
	une moindre mesure.
\end{itemize}
~\\
Ceci confirme bien les résultats attendus, c'est à dire qu'il faut donner la priorité aux produits consommant le moins de ressources.\\
On remarque cependant qu'il est préférable de ne pas abandonner (C) cette fois-ci.
