\newpage
\section{Objectif : Responsable des stocks}
Le responsable des stocks cherche à minimiser le nombre de de produits dans
son stock sous les contraintes définies précedemment.

\subsection{Modélisation}
Soit $Stock(n_{i})$ le nombre de produits en stock (en unités de stock). Cette
fonction est la somme des produits fabriqués à laquelle on soustrait la quantité
de matières premières utilisée.

On a ainsi la formule suivante, où $n_{i}$ est la quantité de produit usiné
(pour chaque produit $i$), et $Q_{i,j}$ est la quantité de matière première par
produit pour chaque produit $i$ et chaque matière première $j$.
 
\begin{equation}
	Stock(n_{i}) = \sum_{i} n_{i} - \sum_{i} n_{i} \times Q_{i,j}
\end{equation}

(formule pas encore valide) 