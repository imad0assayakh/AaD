\newpage
\section{Objectif : Responsable des stocks}
Le responsable des stocks cherche à minimiser le nombre de de produits dans
son stock sous les contraintes définies précedemment.

\subsection{Modélisation}
Soit $\Delta Stock(n_{i})$ la variation du nombre d'unités en stock (l'unité est
inconnue) en fonction du nombre de produits frabriqués. Cette fonction est la
somme des produits fabriqués à laquelle on soustrait la quantité de matières
premières utilisée.

On suppose qu'un produit frabriqué correspond à une unité de stock.

On a ainsi la formule suivante, où $n_{i}$ est la quantité de produit usiné
(pour chaque produit $i$), et $Q_{i,j}$ est la quantité de matière première par
produit pour chaque produit $i$ et chaque matière première $j$.

\begin{equation}
	\Delta Stock(n_{i}) = \sum_{i} n_{i} - n_{i} \times \sum_{j} Q_{i,j}
\end{equation}

$\Delta Stock(n_{i})$ sera une fonction négative ou nulle : en effet, pour
chaque unité de produit frabriqué, au moins une unité de matière première est
consommée. En cherchant à minimiser cette expression, on minimise le stock.

\subsection{Décisions}