\part{Programmation lin�aire multicrit�re}

L'objectif est de trouver une solution de compromis entre les diff�rents responsables.
Pour trouver une telle solution nous serons amen�s � utiliser la programmation multicrit�re (\emph{PLM}).
Auparavant dans la partie 1, nous avons trouv� un optimum pour chaque responsable ind�pendement, ce qui nous conduit � un point de mire. Dans un monde parfait ce point de mire respecterait les contraintes de chaque responsable. Nous devons donc voir si telle est le cas.


\section{Recherche du point de d�part}
Si le point de mire est assez proche de l'ensemble des solutions acceptables, nous choisirons une solution proche de celle d'un responsable.

Sinon, nous allons calculer la satisfaction de chaque objectif, sachant qu'une solution a �t� retenue. Nous devrons alors d�finir des m�triques, correspondant � cette satisfaction. Par exemple, pour le comptable, cette satisfaction sera exprim�e par le ratio du b�n�fice obtenue dans un solution par rapport au b�n�fice maximal.
Ensuite, nous choisirons comme point de d�part la solution qui offre le plus de satisfaction � tout le monde, par exemple en utilisant une moyenne pond�r�e, dont la pond�ration sera bas�e sur \emph{l'importance} de chaque crit�re.

\section{Affinement de la solution}
La solution trouv�e pr�c�demment peut s�rement �tre optimis�e. Il peut �tre interressant de perdre dans un crit�re, si cela nous fait gagner beaucoup dans un autre crit�re, d'autant plus si ce second crit�re est jug� plus \emph{interressant} que le premier.

\section{M�triques utilis�es}
\paragraph{Comptable :}
La m�trique utilis�e sera le pourcentage du b�n�fice par rapport au b�n�fice maximum :
$$
M_{Comptable} = \frac{B_{S}}{B_{max}} \times 100
$$

\paragraph{Responsable d'atelier}
La m�trique utilis�e sera le pourcentage du nombre de produits fabriqu�s par rapport au nombre maximum :
$$
M_{Atelier} = \frac{N_{S}}{N_{max}} \times 100
$$

\paragraph{Responsable des stocks}

