\nTitle{Programmation Linéaire monocritère}

\section{Données}
Soient :
\begin{itemize}
  \item \textbf{T} la matrice des temps unitaires d'usinage d'un produit sur une
  machine (minutes) (\textsl{C.f. Table 1}).
  \item \textbf{Q} la matrice de quantité de matières premières par produit
  (\textsl{C.f. Table 2}).
  \item \textbf{S} la matrice des quantité maximum de matières premières
  (\textsl{C.f. Table 3}).
  \item \textbf{V} la matrice des prix de vente des produits finis (\textsl{C.f.
  Table 4})
  \item \textbf{A} la matrice des prix d'achat des matières premières.
  \item \textbf{C} la matrice des coûts horaires des machines (\textsl{C.f.
  Table 5}).
\end{itemize}

\subsection{Contraintes}
Considérons :
\begin{itemize}
  \item 7 machines $j \in {1, 2, 3, 4, 5, 6 ,7}$
  \item 6 produits $i \in {A, B, C, D, E, F}$
  \item $n_i$ le nombre de d'unités $i$ fabriquées
\end{itemize}
~\\
L'ensemble de la chaine de production est régie par les contraintes suivantes
:\\
\begin{itemize}
  \item \textbf{Le nombre de produits usinés :} Il doit être non nul
  \begin{equation} 
  	\forall i, n_i \ge 0 \label{C0}
  \end{equation}
  
  \item \textbf{La quantité de matières premières :} Elle doit
  être positive.
  \begin{equation} 
  	\forall i, S_i \ge 0 \label{C0}
  \end{equation}
  
  \item \textbf{Le temps d'occupation de chaque machine $i$:} Il doit être
  inférieur au temps de travail
  \begin{equation} 
  	\sum_{j = A}^{F} T_{j,i} . n_j \leq 2.8.60.5 = 4800 \label{C1}
  \end{equation} 
  soit un temps de travail en deux huit, 5 jours par semaine.
  
  \item \textbf{L'utilisation de chaque matière première  $i$:} Elle doit être
  inférieure au stock
  \begin{equation} 
  	\sum_{j = A}^{F} Q_{i,j} . n_j \leq S_i \label{C2}
  \end{equation} 
\end{itemize}

\section{Objectif : Comptable}
Le comptable cherche à maximiser les bénefices sous les contraintes définies
précedemment.

\subsection{Modélisation}
Soit $n_i$ le nombre de produit $i$ fabriqué. Le coup fixe de production
n'influant pas sur notre décision, nous ne considérerons que le coût variable de
production. Il est défini par la formule suivante:
\begin{displaymath}
CV(i) = n_i * \left (\sum_{j = 1}^{7} T_{i,j} .
\frac{C_{i,j}}{60} + \sum_{k = 1}^{3} Q_{k,i} . A_{k} \right )
\end{displaymath}
~\\
Le chiffre d'affaire par produit est :
\begin{displaymath}
CA(i) = n_i . V_i
\end{displaymath}
~\\
Par conséquent le bénefice par produit se calcule de la manière suivante :
\begin{eqnarray*}
	B(i) &=& CA(i) - CV(i)\\
	B(i) &=& n_i * \left (V_i - \sum_{j = 1}^{7} T_{i,j} . \frac{C_{i,j}}{60} +
	\sum_{k = 1}^{3} Q_{k,i} . A_{k} \right )
\end{eqnarray*}
\newpage
\section{Objectif : Responsable d'atelier}
Le responsable d'atelier cherche à maximiser le nombre d'unités (toutes
catégories confondues) produites sous les contraintes définies précedemment.

\subsection{Modélisation}
Soit $N$ le nombre de produits fabriqués.

\begin{equation}
	N = \sum_{i = A}^{F}
\end{equation} 
\newpage
\section{Objectif : Responsable commercial}
Le responsable commercial cherche à équilibrer le nombre
d'unités de ${A, B, C}$ (famille 1) et ${D, E, F}$ (famille 2) afin que ces deux
familles contiennent le même nombre d'unités ( à $\epsilon$
unité(s) près).\\
Autrement dit, l'écart entre le nombre d'unités produite pour la famille A et la famille B doit être inférieur à un seuil $\epsilon$.

\subsection{Modélisation}
Soient :
\begin{itemize}
  \item $N_1$ le nombre de produits de la famille 1 fabriqués.
  \item $N_2$ le nombre de produits de la famille 2 fabriqués.
\end{itemize}

\begin{eqnarray*}
	|N_1 - N_2| &\leq& \epsilon\\
	\Leftrightarrow -\epsilon \leq N_1 - N_2 &\leq& \epsilon\\
	\Leftrightarrow -\epsilon \leq \sum_{i = A}^{C} n_i - \sum_{j = D}^{F} n_j
	&\leq& \epsilon\\
\end{eqnarray*} 
Par concéquent, c'est cette nouvelle contrainte qui, venant s'ajouter aux
contraintes précédentes, va permettre de calculer le nombre d'unités A, B, C,
D, E, et F à fabriquer afin d'équilibrer les deux familles.\\
~\\
Nous obtenons la matrice suivante :
\begin{displaymath}
M = \left(
\begin{array}{cccccc}
1 & 1 & 1 & 1 & 1 & 1\\
\end{array}
\right)
\end{displaymath}
La matrice, très simple, représente la somme des différents produits.\\
La matrice des contraintes devient quant à elle :
INSÉRER MATRICE A MODIFIEE

\subsection{Décisions}

\subsection{Interprétation}
Evidemment, toutes les solutions triviales du type :
\begin{displaymath}
M_p = \left(
\begin{array}{cccccc}
N \pm \epsilon & N \pm \epsilon & N \pm \epsilon & N & N & N\\
\end{array}
\right)
\end{displaymath}
ou encore 
\begin{displaymath}
M_p = \left(
\begin{array}{cccccc}
N & N & N & N \pm \epsilon & N \pm \epsilon & N \pm \epsilon\\
\end{array}
\right)
\end{displaymath}
\begin{center}
\textsl{Où $M_p$ est la matrice du nombre de produit, et $N \in \mathbbm{N}$}
\end{center}
sont des solutions \emph{valables}.\\
~\\
Ceci met en évidence qu'avec les critères définis plus haut, il n'y a pas de
solution plus \og valable\fg ~qu'une autre. Par concéquent nous pouvons :
\begin{itemize}
  \item Choisir une solution au hasard
  \item Augmenter le nombre de critère et notamment ceux en rapport avec les
  stocks disponibles, le prix des matières premières, ou encore le temps
  d'usinage nécessaire.
\end{itemize}
