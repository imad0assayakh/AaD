%%This is a very basic article template.
%%There is just one section and two subsections.
\newpage
\section{Objectif : Responsable commercial}
Le responsable commercial cherche à équilibrer le nombre
d'unités de ${A, B, C}$ (famille 1) et ${D, E, F}$ (famille 2) afin que ces deux
familles contiennent le même nombre d'unités (à $\epsilon$ unité(s) prés).
~\\
En prennant en compte les contraintes définies avant et en ajoutant une nouvelle
contrainte d'équilibre entre les deux familles de produits on se rend compte que
la meilleure solution sera de ne rien produire. C'est solution n'est pas
avantageuse. On introduit par conéquent une autre contrainte - nous allons 
essayer d'équilibrer les familles de produits tout en conservant un bénéfice maximal.

\subsection{Modélisation}
Soient :
 \begin{itemize}
%   \item $N_1$ le nombre de produits de la famille 1 fabriqués.
%   \item $N_2$ le nombre de produits de la famille 2 fabriqués.
	\item $n_A$ le nombre de produits A usinés. 
	\item $n_B$ le nombre de produits B usinés. 
	\item $n_C$ le nombre de produits C usinés. 
	\item $n_D$ le nombre de produits D usinés. 
	\item $n_E$ le nombre de produits E usinés. 
	\item $n_F$ le nombre de produits F usinés. 
\end{itemize}
~\\
Pour étudier l'équilibre entre les deux familles des produits on définira une
fonction qui sera égale à la différence entre les quantités des produits
provenant des deux familles, soit: 

\begin{displaymath}
F = (n_A+n_B+n_C)-(n_D+n_E+n_F)
\end{displaymath}

La contrainte sur l'équilibre va se traduire par l'essai de minimiser la
fonction F, c'est à dire:
\begin{eqnarray*}
	|F| &\leq& \epsilon\\
	\Leftrightarrow -\epsilon \leq F &\leq& \epsilon\\
	\Leftrightarrow -\epsilon \leq (n_A+n_B+n_C)-(n_D+n_E+n_F)
	&\leq& \epsilon\\
	Avec& \epsilon \rightarrow 0.\\
\end{eqnarray*} 
~\\
Ensuite on va essayer de maximiser le bénéfice obtenu. 
On va procéder de la manière suivante:  
\begin{itemize}
   \item Nous fixons un certain bénéfice à atteindre, par
	     exemple 60\% du bénéfice maximal.
   \item Pour le bénéfice choisi, nous essayons de minimiser F. 
\end{itemize}
~\\
En répétant cette démarche un certain nombre de fois, nous pouvons tracer une
courbe représentant la valeur de F en fonction du bénéfice atteint. 
L'interprétation de cette courbe nous permettra de trouver le meilleur
compromis entre nos deux objectifs.

\subsection{Décisions}

