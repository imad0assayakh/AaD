\newpage
\section{Objectif : Comptable}
Le comptable cherche à maximiser les bénefices sous les contraintes définies
précedemment.

\subsection{Modélisation}
Soit $n_i$ le nombre de produit $i$ fabriqué. Le coup fixe de production
n'influant pas sur notre décision, nous ne considérerons que le coût variable de
production. Il est défini par la formule suivante:
\begin{displaymath}
CV(i) = n_i \times \left (\sum_{j = 1}^{7} T_{i,j} \times 
\frac{C_{i,j}}{60} + \sum_{k = 1}^{3} A_{k} \times Q_{k,i} \right )
\end{displaymath}
~\\
Le chiffre d'affaire par produit est :
\begin{displaymath}
CA(i) = n_i \times V_i
\end{displaymath}
~\\
Par conséquent le bénefice par produit se calcule de la manière suivante :
\begin{eqnarray*}
	B(i) &=& CA(i) - CV(i)\\
		&=& n_i  \times \left (V_i - \sum_{j = 1}^{7} T_{i,j} \times \frac{C_{i,j}}{60} +
	\sum_{k = 1}^{3} A_{k} \times Q_{k,i} \right )
\end{eqnarray*}
La matrice permettant de calculer, à partir du vecteur colonne $n$ du nombre de produits sortis de l'usine, le bénéfice d'\emph{Optim} est donc la suivante :
\begin{eqnarray*}
	M_{B} &=&  \left(V - \left( \left(T \times C^{t} \times \frac{1}{60} \right)^{t} + \left(A \times Q\right) \right)\right) \\
		&\simeq& \begin{pmatrix} 6.0667 & 11.9833 & 12.4333 & 9.5333 & 31.6500 & 27.9000 \end{pmatrix}
\end{eqnarray*}

Remarquons qu'elle nous donne explicitement le bénéfice unitaire pour chaque produit. Le produit E est \emph{a priori} le plus intéressant.

Nous chercherons à maximiser la fonction linéaire correspondant à cette matrice, donc (pour prendre une forme plus standard) à minimiser son opposé.


\subsection{Décisions}

En utilisant les outils Matlab, on obtient le résulat suivant :
\begin{lstlisting}
N = linprog(Units, InfEqConstraints, InfEqValues);
\end{lstlisting}

\[
	n_{optimal} = 
\begin{pmatrix}
    0.0000 \\
   20.4082 \\
    0.0000 \\
    0.0000 \\
  242.5000 \\
   94.1837
\end{pmatrix}
\]

\begin{center}
<<<<<<< HEAD
\textbf{Le bénéfice maximum est donc 357.0919 Unités Monaitaires.\\}
~\\
\textit{Lecture : le produit 1 (A) doit être abandonné, le produit 5 (E) doit
être produit en 242,5 exemplaires (242 en entier et un demi exemplaire, terminé la semaine suivante), etc.}
=======
\textbf{Le bénéfice maximum est donc 357.0919 \textcurrency.}
\par\noindent\textit{Lecture : le produit 1 (A) doit être abandonné, le produit 5 (E) doit être produit en 242,5 exemplaires (242 en entier et un demi exemplaire, terminé la semaine suivante), etc.}
>>>>>>> 77f155c849fbbcbf30e915678b3a59b086f78877
\end{center}

Ce résultat était (en partie) prévisible à partir de la matrice $M_{B}$, puisque le produit E est le plus rentable. On devra donc, pour optimiser le bénéfice, en produire le plus possible, tout en utilisant intelligemment les matières premières et ressources humaines restantes pour maximiser le reste du bénéfice. Ainsi, d'après le résultat on préfère fabriquer le produit B au lieu du produit C car il est moins couteux en matière première MP3.

Notons tout de même que ce résultat n'était pas si évident, puisqu'il prend également en compte l'utilisation des ressources (les machines), que nous avons ignorée dans le raisonnement \og intuitif \fg ci-dessus.

